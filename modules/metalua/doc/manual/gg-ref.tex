\section{{\tt gg}, the grammar generator}

\verb|gg| is the grammar generator, the library with which Metalua
parser is built. Knowing it allows you to easily write your own
parsers, and to plug them into mlp, the existing Metalua source
parser. It defines a couple of generators, which take parsers as
parameters, and return a more complex parser as a result by combining
them.

Notice that \verb|gg| sources are thoroughly commented, and try to be
readable as a secondary source of documentation.

There are four main classes in gg, which allow to generate:
\begin{itemize}
\item sequences of keywords and subparsers;
\item keyword-driven sequence sets, i.e. parsers which select a
  sequence parser depending on an initial keyword;
\item lists, i.e. series of an undetermined number of elements of the
  same type, optionnaly separated by a keyword (typically ``{\tt,}'');
\item expressions, built with infix, prefix and suffix operators
  around a primary expression element;
\end{itemize}



\subsection{Sequences}

A sequence parser combines sub-parsers together, calling them one after
the other. A lot of these sub-parsers will simply read a keyword, and
do nothing of it except making sure that it is indeed here: these can
be specified by a simple string. For instance, the following
declarations create parsers that read function declarations, thanks
to some subparsers:
\begin{itemize}
\item \verb|func_stat_name| reads a function name (a list of
  identifiers separated by dots, plus optionally a semicolon and a
  method name);
\item \verb|func_params_content| reads a possibly empty list of
  identifiers separated with commas, plus an optional ``\verb|...|'';
\item \verb|mlp.block| reads a block of statements (here the function
  body);
\item \verb|mlp.id| reads an identifier.
\end{itemize}

\begin{Verbatim}[fontsize=\scriptsize]
-- Read a function definition statement
func_stat = gg.sequence{ "function", func_stat_name, "(",
                         func_params_content, ")", mlp.block, "end" }

-- Read a local function definition statement
func_stat = gg.sequence{ "local", "function", mlp.id, "(",
                         func_params_content, ")", mlp.block, "end" }

\end{Verbatim}

\subsubsection{Constructor {\tt gg.sequence (config\_table) }}

This function returns a sequence parser. \verb|config_table| contains,
in its array part, a sequence of string representing keyword parsers,
and arbitrary sub-parsers. Moreover, the following fields are allowed
in the hash part of the table:

\begin{itemize}
\item\verb|name = <string>|: the parser's name, used to generate error
  messages;
\item\verb|builder = <function>|: if present, whenever the parser is
  called, the list of the sub-parsers results is passed to this
  function, and the function's return value is returned as the
  parser's result. If absent, the list of sub-parser results is simply
  returned. It can be updated at anytime with:\\
  \verb|x.builder = <newval>|.
\item\verb|builder = <string>|: the string is added as a tag to the
  list of sub-parser results. \verb|builder = "foobar"| is equivalent
  to:\\
 \verb|builder = function(x) x.tag="foobar"; return x end|
\item\verb|transformers = <function list>|: applies all the functions
  of the list, in sequence, to the result of the parsing: these
  functions must be of type AST$\rightarrow$AST. For instance, if the
  transformers list is {\tt\{f1, f2, f3\}}, and the builder returns
  {\tt x}, then the whole parser returns {\tt f3(f2(f1(x)))}.
\end{itemize}

\subsubsection{Method {\tt :parse(lexstream)}}

Read a sequence from the lexstream. If the sequence can't be entirely
read, an error occurs. The result is either the list of results of
sub-parsers, or the result of \verb|builder| if it is non-nil. In the
\verb|func_stat| example above, the result would be a list of 3
elements: the results of \verb|func_stat_name|,
\verb|func_params_content| and \verb|mlp.block|.

It can also be directly called as simply \verb|x(lexstream)| instead of
\verb|x:parse(lexstream)|.

\subsubsection{Method {\tt .transformers:add(f)}}
Adds a function at the end of the transformers list.

\subsection{Sequence sets}

In many cases, several sequence parsers can be applied at a given
point, and the choice of the right parser is determined by the next
keyword in the lexstream. This is typically the case for Metalua's
statement parser. All the sequence parsers must start with a keyword
rather than a sub-parser, and that initial keyword must be different
for each sequence parser in the sequence set parser. The sequence set
parser takes care of selecting the appropriate sequence
parser. Moreover, if the next token in the lexstream is not a keyword, or
if it is a keyword but no sequence parser starts with it, the sequence
set parser can have a default parser which is used as a fallback.

For instance, the declaration of \verb|mlp.stat| the Metalua statement
parser looks like:

\begin{verbatim}
mlp.stat = gg.multisequence{
  mlp.do_stat, mlp.while_stat, mlp.repeat_stat, mlp.if_stat... }
\end{verbatim}

\subsubsection{Constructor {\tt gg.multisequence (config\_table)}}


This function returns a sequence set parser. The array part of
\verb|config_table| contains a list of parsers. It also accepts tables
instead of sequence parsers: in this case, these tables are supposed
to be config tables for \verb|gg.sequence| constructor, and are
converted into sequence parsers on-the-fly by calling
\verb|gg.sequence| on them. Each of these sequence parsers has to
start with a keyword, distinct from all other initial keywords,
e.g. it's illegal for two sequences in the same multisequence to start
with keyword {\tt do}; it's also illegal for any parser but the
default one to start with a subparser, e.g. {\tt mlp.id}.

It also accepts the following fields in the table's hash part:
\begin{itemize}
\item\verb|default = <parser>|: if no sequence can be chosen, because
  the next token is not a keyword, or no sequence parser in the set
  starts with that keyword, then the default parser is run instead; if
  no default parser is provided and no sequence parser can be chosen,
  an error is generated when parsing.
\item\verb|name = <string>|: the parser's name, used to generate arror
  messages;
\item\verb|builder = <function>|: if present, whenever the parser is
  called, the selected parser's result is passed to this
  function, and the function's return value is returned as the
  parser's result. If absent, the selected parser's result is simply
  returned. It can be updated at anytime with
  \verb|x.builder = <newval>|.
\item\verb|builder = <string>|: the string is added as a tag to the
  list of sub-parser results. \verb|builder = "foobar"| is equivalent
  to:\\
 \verb|builder = function(x) return { tag = "foobar"; unpack (x) } end|
\item\verb|transformers = <function list>|: applies all the functions
  of the list, in sequence, to the result of the parsing: these
  functions must be of type AST$\rightarrow$AST.
\end{itemize}

\subsubsection{Method {\tt :parse(lexstream)}}

Read from the lexstream. The result returned is the result of the selected
parser (one of the sequence parsers, or the default parser). That
result is either returned directly, or passed through \verb|builder|
if this field is non-nil.

It can also be directly called as simply \verb|x(lexstream)| instead of
\verb|x:parse(lexstream)|.

\subsubsection{Method {\tt :add(sequence\_parser)}}

Take a sequence parser, or a config table that would be accepted by
\verb|gg.sequence| to build a sequence parser. Add that parser to the
set of sequence parsers handled by x. Cause an error if the parser
doesn't start with a keyword, or if that initial keyword is already
reserved by a registered sequence parser, or if the parser is not a
sequence parser.

\subsubsection{Field {\tt .default}}
This field contains the default parser, and can be set to another
parser at any time.

\subsubsection{Method {\tt .transformers:add}}
Adds a function at the end of the transformers list.

\subsubsection{Method {\tt :get(keyword)}}
Takes a keyword (as a string), and returns the sequence in the set
starting with that keyword, or nil if there is no such sequence.

\subsubsection{Method {\tt :del(keyword)}}
Removes the sequence parser starting with keyword {\tt kw}.

\subsection{List parser}

Sequence parsers allow to chain several different sub-parser. Another
common need is to read a series of identical elements into a list, but
without knowing in advance how many of such elements will be
found. This allows to read lists of arguments in a call, lists of
parameters in a function definition, lists of statements in a
block\ldots

Another common feature of such lists is that elements of the list are
separated by keywords, typically semicolons or commas. These are
handled by the list parser generator.

A list parser needs a way to know when the list is finished. When
elements are separated by keyword separators, this is easy to
determine: the list stops when an element is not followed by a
separator. But two cases remain unsolved:
\begin{itemize}
\item When a list is allowed to be empty, no separator keyword allows
  the parser to realize that it is in front of an empty list: it would
  call the element parser, and that parser would fail;
\item when there are no separator keyword separator specified, they
  can't be used to determine the end of the list.
\end{itemize}

For these two cases, a list parser can specify a set of terminator
keywords: in separator-less lists, the parser returns as soon as a 
terminator keyword is found where an element would otherwise have been
read. In lists with separators, if terminators are specified, and such
a terminator is found at the beginning of the list, then no element is
parsed, and an empty list is returned. For instance, for argument
lists, ``\verb|)|'' would be specified as a terminator, so that empty
argument lists ``\verb|()|'' are handled properly.

Beware that separators are consumed from the lexstream stream, but
terminators are not.

%\caveat{FIXME: check that it works as advertized (for list parserd
%  with separators AND terminators; it's related to the trailing
%  separator issue in block and table\_content parsers). There still is
%  a design issue to settle here.}

\subsubsection{Constructor {\tt gg.list (config\_table)}}

This function returns a list parser. \verb|config_table| can contain
the following fields:
\begin{itemize}

\item\verb|primary = <parser>| (mandatory): the parser used to read
  elemetns of the list;

\item\verb|separators = <list> |: list of strings representing the
  keywords accepted as element separators. If only one separator is
  allowed, then the string can be passed outside the list:\\
  \verb|separators = "foo"| is the same as 
  \verb|separators = { "foo" }|.

\item\verb|terminators = <list> |: list of strings representing the
  keywords accepted as list terminators. If only one separator is
  allowed, then the string can be passed outside the list.

\item\verb|name = <string>|: the parser's name, used to generate arror
  messages;

\item\verb|builder = <function>|: if present, whenever the parser is
  called, the list of primary parser results is passed to this
  function, and the function's return value is returned as the
  parser's result. If absent, the list of sub-parser results is simply
  returned. It can be updated at anytime with
  \verb|x.builder = <newval>|.

\item\verb|builder = <string>|: the string is added as a tag to the
  list of sub-parser results. \verb|builder = "foobar"| is equivalent
  to:\\
 \verb|builder = function(x) return { tag = "foobar"; unpack (x) } end|

\item\verb|transformers = <function list>|: applies all the functions
  of the list, in sequence, to the result of the parsing: these
  functions must be of type AST$\rightarrow$AST.

\item if keyless element is found in \verb|config_table|, and there is
  no \verb|primary| key in the table, then it is expected to be a
  parser, and it is considered to be the primary parser.
\end{itemize}

\subsubsection{Method {\tt :parse (lexstream)}}

Read a list from the lexstream. The result is either the list of elements
as read by the primary parser, or the result of that list passed
through \verb|builder| if it is specified.

It can also be directly called as simply \verb|x(lexstream)| instead of
\verb|x:parse(lexstream)|.

\subsubsection{Method {\tt .transformers:add}}
Adds a function at the end of the transformers list.

\subsection{Method {\tt .separators:add}}
Adds a string to the list of separators.

\subsection{Method {\tt .terminators:add}}
Adds a string to the list of terminators.

\subsection{Expression parser}

This is a very powerfull parser generator, but it ensues that its API
is quite large. An expression parser relies on a primary parser, and
the elements read by this parser can be:
\begin{itemize}
\item combined pairwise by infix operators;
\item modified by prefix operators;
\item modified by suffix operators.
\end{itemize}

All operators are described in a way analoguous to sequence config
tables: a sequence of keywords-as-strings and subparsers in a table,
plus the usual \verb|builder| and \verb|transformers|
fields. Each kind of operator has its own signature for \verb|builder|
functions, and some specific additional information such as precedence
or associativity. As in multisequences, the choice among operators is
determined by the initial keyword. Therefore, it is illegal to have
two operator sequences which start with the same keyword (for
instance, two infix sequences starting with keyword ``\$'').

Most of the time, the sequences representing operators will have a
single keyword, and no subparser. For instance, the addition is
represented as:\\
\verb~{ "+",  prec=60, assoc="left", builder= |a, _, b| `Op{ `Add, a, b } }~


\paragraph{Infix operators}
Infix operators are described by a table whose array-part works as for
sequence parsers. Besides this array-part and the usual {\tt
  transformers} list, the table accepts the following fields in its hash-part:
\begin{itemize}

\item\verb|prec = <number>| its precedence. The higher the precedence, 
  the tighter the operator bind with respect to other operators. For
  instance, in Metalua, addition precedence is 60, whereas
  multiplication precedence is 70.

\item\verb|assoc = <string>| is one of \verb|"left"|, \verb|"right"|,
  \verb|"flat"| or \verb|"none"|, and specifies how an operator
  associates. If not specified, the default associativity is
  \verb|"left"|. 

  Left and right describe how to read sequences of operators with the
  same precedence, e.g. addition is left associative ({\tt 1+2+3} reads as
  {\tt(1+2)+3}), whereas exponentiation is right-associative (\verb|1^2^3|
  reads as \verb|1^(2^3)|).

  If an operator is non-associative and an ambiguity is found, a
  parsing error occurs. 

  Finally, flat operators get series of them collected in a list,
  which is passed to the corresponding builder as a single
  parameter. For instance, if \verb|++| is declared as flat and its
  builder is \verb|f|, then whenever {\tt 1++2++3++4} is parsed, the
  result returned is {\tt f\{1, 2, 3, 4\}}.

\item\verb|builder = <function>| the usual result transformer. The
  function takes as parameters the left operand, the result of the
  sequence parser (i.e. \verb|{ }| if the sequence contains no
  subparser), and the right operand; it must return the resulting AST.

\end{itemize}

%For instance, Metalua's addition parser could be described by:

%\begin{verbatim}
%{ "+", assoc = "left", prec = 60, builder = |a,_,b| `Op{ `Add, a, b } }
%\end{verbatim}

\paragraph{Prefix operators}
These operators are placed before the sub-expression they modify. They
have the same properties as infix operators, except that they don't
have an \verb|assoc| field, and \verb|builder| takes {\tt|operator,
  operand|} instead of {\tt|left\_operand, operator, right\_operand|}.

\paragraph{Suffix operators}
Same as prefix operators, except that \verb|builder| takes
{\tt|operand, operator|} instead of {\tt|operator, operand|}.

\subsubsection{Constructor {\tt gg.expr (config\_table)}}

This function returns an expression parser. \verb|config_table|
is a table of fields which describes the kind of expression to be
read by the parser. The following fields can appear in the table:

\begin{itemize}
\item\verb|primary| (mandatory): the primary parser, which reads the
  primary elements linked by operators. In Metalua expression parsers,
  that would be numbers, strings, identifiers\ldots It is often a
  multisequence parser, although that's not mandatory.
\item\verb|prefix|: a list of tables representing prefix operator
  sequences, as described above. It supports a {\tt default} parser:
  this parser is considered to have succesfully parsed a prefix
  operator if it returns a non-false result.
\item\verb|infix|: a list of tables representing infix operator
  sequences, as described above. Supports a {\tt default} parser.
\item\verb|suffix|: a list of tables representing suffix operator
  sequences, as described above. Supports a {\tt default} parser.
\end{itemize}

\subsubsection{Methods {\tt .prefix:add()}, {\tt .infix:add()}, {\tt
    .suffix:add()}}
Add an operator in the relevant table. The argument must be an
operator sequence table, as described above.

\subsubsection{Method {\tt :add()}}
This is just a shortcut for {\tt primary.add}. Unspecified behavior if
{\tt primary} doesn't support method {\tt add}.


\subsubsection{Method {\tt :parse (lexstream)}}
Read a list from the lexstream. The result is built by \verb|builder1| calls.

It can also be directly called as simply \verb|x(lexstream)| instead of
\verb|x:parse(lexstream)|.

\subsubsection{Method {\tt :tostring()}}

Returns a string representing the parser. Mainly useful for error
message generation.

\subsection{{\tt onkeyword} parser}

Takes a list of keywords and a parser: if the next token is one of the
keywords in the list, runs the parser; if not, simply returns
\verb|false|.

Notice that by default, the keyword is consumed by the
\verb|onkeyword| parser. If you want it not to be consumed, but
instead passed to the internal parser, add a \verb|peek=true| entry in
the config table.

\subsubsection{Constructor {\tt gg.onkeyword (config\_table)}}

Create a keyword-conditionnal parser. \verb|config_table| can contain:

\begin{itemize}
\item strings, representing the triggerring keywords (at least one);
\item the parser to run if one of the keywords is found (exactly one);
\item \verb|peek=<boolean>| to indicate whether recognized keywords
  must be consumed or passed to the inner parser.
\end{itemize}
The order of elements in the list is not relevant.

\subsubsection{Method {\tt :parse (lexstream)}}

Run the parser. The result is the internal parser's result, or
\verb|false| if the next token in the lexstream wasn't one of the
specified keywords.

It can also be directly called as simply \verb|x(lexstream)| instead of
\verb|x:parse(lexstream)|.

\subsection{{\tt optkeyword} parser}

Watch for optional keywords: an \verb|optkeyword| parser has a list of
keyword strings as a configuration. If such a keyword is found as the
nex lexstream element upon parsing, the keyword is consumed and that
string is returned. If not, \verb|false| is returned.

\subsubsection{Constructor {\tt gg.optkeyword (keyword1, keyword2, ...)}}

Return a \verb|gg.optkeyword| parser, which accepts all of the
keywords given as parameters, and returns either the found keyword, or
\verb|false| if none is found.
